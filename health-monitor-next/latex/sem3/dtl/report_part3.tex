\pagestyle{fancy}
\thispagestyle{fancy}
\chapter{Define}
\section{Introduction to problem definition}

After gathering insights from our users through surveys and interviews, we needed to clearly define the problems we were trying to solve. While users shared many challenges, we focused on identifying the core issues that we could realistically address.

Many users expressed frustration with managing their health information across different platforms and understanding their medical data. We needed to find a balance between what users wanted and what we could technically achieve while ensuring data privacy and security.

The key was to strip away complex features and focus on the fundamental problems: How do people manage their health information? What makes it difficult? What solutions would actually help them?

\section{Problem Statement and Point of View}

\subsection{Problem Statement}
Key issues identified from our research:
\begin{itemize}
    \item Managing health records across multiple platforms is time-consuming and confusing
    \item Understanding medical terminology and test results is challenging for most users
    \item Tracking real-time health metrics and connecting them to historical data is difficult
    \item Sharing health information securely with healthcare providers is complicated
\end{itemize}

\subsection{Point of View (POV)}
\begin{itemize}
    \item \textbf{User:}\\
    Health-conscious individuals who want to actively manage their health data but struggle with current fragmented solutions.

    \item \textbf{Need:}\\
    A unified, easy-to-use platform that helps them understand and manage their health information effectively.

    \item \textbf{Insight:}\\
    Users want to take control of their health data but need help interpreting and organizing it in a meaningful way.

    \item \textbf{Define:}\\
    Creating a system that simplifies health data management while making the information more accessible and understandable.
\end{itemize}

\section{How might we questions}

Our "How Might We" (HMW) questions served as catalysts for brainstorming, striking a balance between broad possibility and focused direction. We identified six key questions that encompassed the project's scope:

\begin{enumerate}
    \item \textbf{How might we centralize personal health data while ensuring privacy?}\\
    This question addresses the challenge of fragmented health information across multiple platforms while maintaining strict data protection standards.

    \item \textbf{How might we make health data more understandable to users?}\\
    Focusing on translating complex medical information into accessible insights that users can act upon.

    \item \textbf{How might we enable real-time health monitoring without being intrusive?}\\
    Addressing the balance between continuous health tracking and user comfort/privacy.

    \item \textbf{How might we integrate various health data sources seamlessly?}\\
    Looking at ways to combine data from different devices and platforms into a unified view.

    \item \textbf{How might we provide AI assistance while maintaining user trust?}\\
    Exploring the balance between automated insights and transparency in AI decision-making.

    \item \textbf{How might we ensure data security while enabling easy access?}\\
    Addressing the challenge of making health data readily available while protecting sensitive information.
\end{enumerate}

\section{Design thinking challenge/s identified}

Our final Point of View statement is: Users need a way to manage and understand their health data because fragmented information and complex medical terminology result in reduced health awareness and delayed interventions.

\subsection{Challenges identified for Users:}
\begin{itemize}
    \item \textbf{Data Fragmentation:} Health records scattered across multiple platforms and providers
    \item \textbf{Privacy Concerns:} Fear of sensitive health data being exposed or misused
    \item \textbf{Technical Barriers:} Difficulty in setting up and using health monitoring devices
    \item \textbf{Information Overload:} Overwhelming amount of health data without clear insights
    \item \textbf{Limited Integration:} Lack of connection between different health monitoring tools
    \item \textbf{Understanding Gap:} Difficulty interpreting medical terminology and test results
    \item \textbf{Access Issues:} Problems retrieving specific health information when needed
    \item \textbf{Trust in Technology:} Concerns about AI reliability in health monitoring
\end{itemize}

\subsection{Challenges identified for Healthcare Providers:}
\begin{itemize}
    \item \textbf{Data Consistency:} Ensuring accurate and standardized health information
    \item \textbf{Integration Issues:} Difficulty accessing patient-collected health data
    \item \textbf{Privacy Compliance:} Meeting regulatory requirements while sharing data
    \item \textbf{Real-time Monitoring:} Accessing up-to-date patient health metrics
    \item \textbf{Data Verification:} Validating the accuracy of user-reported health data
    \item \textbf{System Compatibility:} Integrating with existing healthcare systems
    \item \textbf{Patient Communication:} Efficient sharing of health insights and recommendations
    \item \textbf{Data Security:} Protecting sensitive health information during transmission
\end{itemize}

\subsection{SCAMPER Method}
SCAMPER is a structured way of looking at different aspects of the problem:
\begin{itemize}
    \item Substitute: Replace manual data entry with automated data collection from IoT sensors to enhance accuracy and efficiency.
    \item Combine: Integrate real-time sensor data and AI analytics to deliver personalized health insights and recommendations.
    \item Adapt: Tailor existing AI models to generate customized health advice based on individual user data inputs.
\end{itemize}

inputs:
\begin{itemize}
    \item Modify: Enhance the RAG pipeline by incorporating diverse data sources, including wearables and mobile health apps.
    \item Put to Another Use: Utilize AI capabilities for community health insights, aiding public health officials in tracking trends.
    \item Eliminate: Streamline the user onboarding process by removing redundant steps and focusing on essential features.
    \item Reverse: Empower users to define their health goals first, allowing AI analysis to be driven by personal preferences rather than raw data.
\end{itemize}

\begin{figure}[H]
    \centering
    \includegraphics[width=1.0\textwidth]{figures/scamper_model.png}
    \caption{Figure 3.2: SCAMPER Model outline}
\end{figure}

\subsection{How Might We?}
\begin{enumerate}
    \item How might we encourage users to monitor their vital health metrics more consistently for better health insights?
    \item How might we design an AI assistant that can respond to the user's queries using medical records and sensor data?
    \item How might we create a more efficient and secure way for users to store and manage their health records?
    \item How might we identify and integrate the most valuable features in a health monitoring platform to enhance user experience?
    \item How might we increase user trust in AI's ability to deliver reliable health insights and predictions?
    \item How might we make it easier for patients to track their most frequently needed types of medical documents?
    \item How might we enable automatic extraction of key information from patients' medical documents?
\end{enumerate} 