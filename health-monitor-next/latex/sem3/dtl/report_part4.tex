\pagestyle{fancy}
\thispagestyle{fancy}
\chapter{Ideate}
\section{Introduction to ideation}

Ideation is an extremely creative process where designers generate ideas in sessions. It accounts as the third stage in the Design Thinking Process. We approached this phase with open minds to produce as many ideas as possible to address the problem statement. The key to this phase was maintaining a judge-free, positive and supportive environment.

It is very difficult to simulate a prejudice-free environment when we ourselves have preconceived notions. First, we needed to unlearn the process of immediate opinion formation and have a third-person perspective that is completely free of judgements. This concept gave rise to numerous innovative ideas for brainstorming. Sometimes even when the best minds merge, there may be ideas that are audience favorites and some only humor the one who owns it. Either way, emotions must be kept aside in terms of whose contribution is being recognized.

\section{Ideation technique/s used and description}

There are a variety of ideation techniques that can be used for generation of ideas. Ideation techniques help in creative flow and help in generation of a variety of ideas. Not all of these ideas may be good, but every idea in turn could lead to a better idea, and hence none of them can be neglected.

For generating ideas, we employed some well-known ideation techniques, which have been listed below. By employing these techniques, by the end of our ideation process, we were able to come up with over 80 ideas! The techniques used by us for generation of ideas include:

\subsection{Brainstorming Cloud}
Brainstorming refers to a group of people coming up with ideas and writing them down. It serves as a brain dump.

\begin{figure}[H]
    \centering
    \includegraphics[width=0.8\textwidth]{figures/brainstorm_cloud.png}
    \caption{Brainstorming Cloud for Health Data Management}
\end{figure}

\subsection{SCAMPER Method}
SCAMPER is a structured way of looking at different aspects of the problem:
\begin{itemize}
    \item Substitute: Replace manual data entry with automated data collection from IoT sensors to enhance accuracy and efficiency.
    \item Combine: Integrate real-time sensor data and AI analytics to deliver personalized health insights and recommendations.
    \item Adapt: Tailor existing AI models to generate customized health advice based on individual user data inputs.
\end{itemize}

inputs:
\begin{itemize}
    \item Modify: Enhance the RAG pipeline by incorporating diverse data sources, including wearables and mobile health apps.
    \item Put to Another Use: Utilize AI capabilities for community health insights, aiding public health officials in tracking trends.
    \item Eliminate: Streamline the user onboarding process by removing redundant steps and focusing on essential features.
    \item Reverse: Empower users to define their health goals first, allowing AI analysis to be driven by personal preferences rather than raw data.
\end{itemize}

\begin{figure}[H]
    \centering
    \includegraphics[width=1.0\textwidth]{figures/scamper_model.png}
    \caption{Figure 4.2: SCAMPER Model outline}
\end{figure}

\subsection{How Might We?}
\begin{enumerate}
    \item How might we encourage users to monitor their vital health metrics more consistently for better health insights?
    \item How might we design an AI assistant that can respond to the user's queries using medical records and sensor data?
    \item How might we create a more efficient and secure way for users to store and manage their health records?
    \item How might we identify and integrate the most valuable features in a health monitoring platform to enhance user experience?
    \item How might we increase user trust in AI's ability to deliver reliable health insights and predictions?
    \item How might we make it easier for patients to track their most frequently needed types of medical documents?
    \item How might we enable automatic extraction of key information from patients' medical documents?
\end{enumerate}

\section{Ideas Generated}
These techniques were very helpful and allowed us to think differently and creatively. By employing the above ideation techniques, we were able to generate 80+ ideas, which we've categorized below:

\subsection{Data Management Ideas}
Ideas focused on efficient health data organization:
\begin{itemize}
    \item Unified dashboard for all health metrics
    \item One-click health record upload
    \item Automated document categorization
    \item Cloud-based secure storage system
\end{itemize}

\subsection{User Interface Ideas}
Ideas improving user interaction and experience:
\begin{itemize}
    \item Natural language search functionality
    \item Intuitive health data visualization
    \item Simplified medical terminology display
    \item Customizable dashboard layouts
\end{itemize}

\subsection{AI and Analytics Ideas}
Ideas leveraging artificial intelligence:
\begin{itemize}
    \item AI-powered health assistant
    \item Automated health report analysis
    \item Predictive health insights
    \item Smart medication reminders
\end{itemize}

\subsection{Integration Ideas}
Ideas for comprehensive health monitoring:
\begin{itemize}
    \item IoT sensor data integration
    \item Real-time health metric tracking
    \item Healthcare provider connectivity
    \item Emergency alert system
\end{itemize}

\section{Implementation Strategy}
Based on our ideation results, we developed a phased implementation approach:

\begin{itemize}
    \item Phase 1: Core platform development and data integration
    \item Phase 2: AI assistant and natural language processing
    \item Phase 3: Creating the agentic-RAG pipeline and real-time monitoring
    \item Phase 4: Advanced analytics using RAG and NLP and reporting features
\end{itemize}

\begin{figure}[H]
    \centering
    \includegraphics[width=0.7\textwidth]{figures/implementation_methodology.png}
    \caption{Figure 4.3: Implementation Methodology}
\end{figure} 