\section{System Design and Implementation}

\subsection{Requirements Specification}

\subsubsection{Functional Requirements}
\begin{enumerate}
    \item \textbf{Health Data Management}
    \begin{itemize}
        \item Upload and process medical records
        \item Store and retrieve health data securely
        \item Manage user health profiles
    \end{itemize}

    \item \textbf{AI-Powered Chat System}
    \begin{itemize}
        \item Process natural language queries
        \item Provide context-aware responses
        \item Access and utilize medical knowledge base
    \end{itemize}

    \item \textbf{Real-time Health Monitoring}
    \begin{itemize}
        \item Collect sensor data (heart rate, temperature, SpO2)
        \item Display real-time health metrics
        \item Store historical health data
    \end{itemize}
\end{enumerate}

\subsubsection{Non-Functional Requirements}
\begin{enumerate}
    \item \textbf{Performance}
    \begin{itemize}
        \item Response time < 2 seconds for queries
        \item Real-time data updates < 100ms
        \item Support multiple concurrent users
    \end{itemize}

    \item \textbf{Security}
    \begin{itemize}
        \item Secure data transmission
        \item User authentication and authorization
        \item HIPAA compliance considerations
    \end{itemize}

    \item \textbf{Reliability}
    \begin{itemize}
        \item System availability > 99.9%
        \item Data backup and recovery
        \item Error handling and logging
    \end{itemize}
\end{enumerate}

\subsection{System Architecture}
The system follows a modern microservices architecture:

\begin{enumerate}
    \item \textbf{Frontend Layer}
    \begin{itemize}
        \item Next.js 14 application with App Router
        \item TypeScript for type safety
        \item Tailwind CSS and shadcn/ui for UI components
        \item TanStack Query for data fetching
    \end{itemize}

    \item \textbf{Backend Services}
    \begin{itemize}
        \item FastAPI service for RAG pipeline
        \item Supabase for real-time data and authentication
        \item PostgreSQL with pgvector for vector storage
    \end{itemize}

    \item \textbf{External Services}
    \begin{itemize}
        \item Cohere for embedding generation
        \item Arduino IoT platform for sensor data
    \end{itemize}
\end{enumerate}

\subsection{RAG Implementation}
The RAG pipeline is implemented as follows:

\begin{enumerate}
    \item \textbf{Document Processing}
    \begin{itemize}
        \item Medical records are uploaded and processed
        \item Text is extracted and chunked
        \item Cohere generates embeddings
        \item Vectors stored in Supabase pgvector
    \end{itemize}

    \item \textbf{Query Processing}
    \begin{itemize}
        \item User queries processed through Next.js API
        \item FastAPI backend handles embedding generation
        \item Vector similarity search retrieves context
        \item Response generated using retrieved context
    \end{itemize}
\end{enumerate}

\subsection{Real-time Health Monitoring}
The health monitoring system consists of:

\begin{enumerate}
    \item \textbf{Hardware Layer}
    \begin{itemize}
        \item Arduino UNO with ESP8266 WiFi module
        \item Multiple health sensors (Heart rate, Temperature, SpO2)
        \item Real-time data transmission setup
    \end{itemize}

    \item \textbf{Data Integration}
    \begin{itemize}
        \item Supabase real-time subscriptions
        \item Live data visualization using Recharts
        \item Automatic data synchronization
    \end{itemize}
\end{enumerate}

\begin{figure}[H]
    \centering
    \includegraphics[width=0.8\textwidth]{images/system_architecture.png}
    \caption{Overall System Architecture}
\end{figure}

\subsection{Key Features Implementation}

\subsubsection{Health Data Management}
\begin{itemize}
    \item Secure medical record storage
    \item Real-time sensor data integration
    \item Historical data analysis
\end{itemize}

\subsubsection{AI-Powered Chat}
\begin{itemize}
    \item Context-aware responses
    \item Medical knowledge integration
    \item Personalized health insights
\end{itemize}

\subsubsection{Real-time Monitoring}
\begin{itemize}
    \item Live health metrics visualization
    \item Automated data collection
    \item Trend analysis and reporting
\end{itemize}

\begin{figure}[H]
    \centering
    \includegraphics[width=0.8\textwidth]{images/feature_implementation.png}
    \caption{Feature Implementation Overview}
\end{figure}

\subsection{Implementation Details}

\subsubsection{Frontend Components}
Example of a React component using TanStack Query:
\begin{lstlisting}[language=typescript,
                   basicstyle=\ttfamily\small,
                   keywordstyle=\color{blue},
                   stringstyle=\color{red},
                   commentstyle=\color{green!60!black}]
// app/health-records/page.tsx
export default function HealthRecords() {
    const { data, isLoading } = useQuery({
        queryKey: ['health-records'],
        queryFn: fetchHealthRecords,
    })

    if (isLoading) return <LoadingSpinner />

    return (
        <div className="grid grid-cols-1 md:grid-cols-2 gap-4">
            {data?.map((record) => (
                <RecordCard key={record.id} record={record} />
            ))}
        </div>
    )
}
\end{lstlisting}

\subsubsection{Backend Services}
Example of FastAPI endpoint and vector search:
\begin{lstlisting}[language=python,
                   basicstyle=\ttfamily\small,
                   keywordstyle=\color{blue},
                   stringstyle=\color{red},
                   commentstyle=\color{green!60!black}]
@router.post("/rag-query")
async def rag_query(request: RAGQueryRequest):
    try:
        query_embedding = generate_embedding(request.query)
        search_results = query_embeddings(query_embedding)
        context = "\n".join([
            result['text_content'] 
            for result in search_results
        ])
        return {
            "query": request.query,
            "context": context,
            "response": f"Response based on context: {context}"
        }
    except Exception as e:
        raise HTTPException(status_code=500, detail=str(e))
\end{lstlisting}

\subsubsection{Vector Search Implementation}
PostgreSQL function for vector similarity search:
\begin{lstlisting}[language=sql,
                   basicstyle=\ttfamily\small,
                   keywordstyle=\color{blue},
                   stringstyle=\color{red},
                   commentstyle=\color{green!60!black}]
CREATE OR REPLACE FUNCTION match_documents (
    query_embedding vector(1536),
    match_threshold float,
    match_count int
)
RETURNS TABLE (
    id bigint,
    content text,
    similarity float
)
LANGUAGE plpgsql
AS $$
BEGIN
    RETURN QUERY
    SELECT
        documents.id,
        documents.content,
        1 - (documents.embedding <=> query_embedding) AS similarity
    FROM documents
    WHERE 1 - (documents.embedding <=> query_embedding) > match_threshold
    ORDER BY similarity DESC
    LIMIT match_count;
END;
$$;
\end{lstlisting}

\subsubsection{RAG Pipeline Implementation}
The RAG system is implemented through a Next.js API route that communicates with our FastAPI backend:

\begin{lstlisting}[language=typescript,
                   basicstyle=\ttfamily\small,
                   keywordstyle=\color{blue},
                   stringstyle=\color{red},
                   commentstyle=\color{green!60!black}]
// app/api/rag-query/route.ts
export async function POST(req: NextRequest) {
  const body = await req.json();
  const { query, llm_choice, match_count, user_id } = body;

  try {
    const response = await fetch(`${API_URL}/rag-query-v2`, {
      method: 'POST',
      headers: {
        'Content-Type': 'application/json',
      },
      body: JSON.stringify({ 
        query, 
        llm_choice, 
        match_count, 
        user_id
      }),
    });

    if (!response.ok) {
      throw new Error('Failed to get response from RAG query');
    }

    const data = await response.json();
    return NextResponse.json(data);
  } catch (error) {
    return NextResponse.json(
      { error: 'An error occurred during the RAG query' }, 
      { status: 500 }
    );
  }
}
\end{lstlisting}

\subsubsection{Real-time Health Monitoring}
The health monitoring dashboard implements real-time data visualization using Supabase subscriptions:

\begin{lstlisting}[language=typescript,
                   basicstyle=\ttfamily\small,
                   keywordstyle=\color{blue},
                   stringstyle=\color{red},
                   commentstyle=\color{green!60!black}]
// app/(app)/health/monitor/page.tsx
export default function HealthMonitor() {
  const [healthData, setHealthData] = useState<HealthData[]>([]);
  const [isStreaming, setIsStreaming] = useState(false);
  const supabase = createSupabaseBrowser();

  useEffect(() => {
    const channel = supabase
      .channel('sensor_data')
      .on(
        'postgres_changes',
        {
          event: 'INSERT',
          schema: 'public',
          table: 'sensor_data',
        },
        (payload) => {
          if (isStreaming) {
            setHealthData(prevData => {
              const newData = [...prevData, payload.new as HealthData];
              return newData.slice(-100); // Keep last 100 records
            });
          }
        }
      )
      .subscribe();

    return () => {
      supabase.removeChannel(channel);
    };
  }, [supabase, isStreaming]);

  return (
    <div className="grid grid-cols-1 md:grid-cols-2 gap-4">
      <HealthChart 
        data={healthData} 
        dataKey="beat_avg" 
        title="Heart Rate" 
        unit="bpm" 
        chartType={chartType} 
      />
      {/* Other charts */}
    </div>
  );
}
\end{lstlisting}

\subsubsection{Health Data Visualization}
Implementation of the chart component for health metrics:

\begin{lstlisting}[language=typescript,
                   basicstyle=\ttfamily\small,
                   keywordstyle=\color{blue},
                   stringstyle=\color{red},
                   commentstyle=\color{green!60!black}]
// app/(app)/health/monitor/components/HealthChart.tsx
export function HealthChart({ 
  data, 
  dataKey, 
  title, 
  unit, 
  chartType 
}: HealthChartProps) {
  const chartConfig = {
    [dataKey]: {
      label: title,
      color: "hsl(var(--chart-1))",
    },
    // ... other configurations
  } satisfies ChartConfig

  return (
    <Card>
      <CardHeader>
        <CardTitle>{title}</CardTitle>
        <CardDescription>
          Real-time {title.toLowerCase()} measurements
        </CardDescription>
      </CardHeader>
      <CardContent>
        <ChartContainer config={chartConfig}>
          {chartType === 'area' ? (
            <AreaChart {...commonChartProps}>
              <CartesianGrid vertical={false} strokeDasharray="3 3" />
              <XAxis
                dataKey="created_at"
                tickFormatter={formatXAxis}
                interval="preserveStartEnd"
                minTickGap={50}
              />
              <YAxis />
              <ChartTooltip content={<CustomTooltip />} />
              <Area
                {...commonDataComponentProps}
                fill={chartConfig[dataKey].color}
              />
            </AreaChart>
          ) : (
            // Line chart implementation
          )}
        </ChartContainer>
      </CardContent>
    </Card>
  )
}
\end{lstlisting} 