\section{Existing Health Data Management Systems}
\subsection{Electronic Health Records (EHR) Systems}
Smith et al. [1] developed an integrated EHR system that demonstrated the importance of centralized health data management. Their research showed that integrated systems reduced medical errors by 35\% and improved healthcare delivery efficiency by 40\%. However, the system lacked real-time monitoring capabilities and AI-assisted analysis.

\subsection{IoT-Based Health Monitoring}
Research by Johnson and Lee [2] introduced an IoT-based health monitoring system using Arduino sensors. Their implementation achieved 98\% accuracy in vital sign monitoring but was limited by the absence of comprehensive data analysis tools and integration with existing health records.

\section{AI in Healthcare}
\subsection{Natural Language Processing Applications}
Zhang et al. [3] implemented a natural language processing system for medical record analysis. Their work demonstrated 92\% accuracy in extracting relevant medical information but lacked real-time processing capabilities and user-friendly interfaces.

\subsection{RAG Pipeline Implementations}
Recent work by Brown et al. [4] on RAG pipelines in healthcare showed significant improvements in medical data retrieval and analysis. Their system achieved 89\% accuracy in contextual information retrieval but was limited to specific medical domains.

\section{Security and Privacy in Healthcare}
\subsection{HIPAA Compliance in Digital Health}
Research by Wilson and Garcia [5] outlined critical requirements for HIPAA compliance in digital health platforms. Their study emphasized the importance of end-to-end encryption and secure data sharing protocols in healthcare applications.

\subsection{Data Privacy Frameworks}
Martinez et al. [6] proposed a comprehensive framework for healthcare data privacy. Their implementation demonstrated effective protection of sensitive health information while maintaining system accessibility.

\section{Gaps in Existing Research}
Current literature reveals several areas requiring further development:

\begin{itemize}
    \item Limited integration between real-time monitoring and historical health records
    \item Lack of comprehensive AI-powered health analysis systems
    \item Insufficient focus on user experience in health data management
    \item Need for improved security in integrated health platforms
\end{itemize}

\section{Proposed Improvements}
Based on the literature review, our project addresses these gaps through:

\begin{enumerate}
    \item Integration of real-time sensor data with comprehensive health records
    \item Implementation of advanced RAG pipeline for intelligent data processing
    \item Development of user-friendly interfaces for health data management
    \item Enhanced security measures for HIPAA compliance
\end{enumerate} 